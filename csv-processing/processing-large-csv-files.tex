\documentclass[11pt]{article}
\usepackage[fntef,hyperref,UTF8]{ctexcap}
\usepackage[a4paper, margin=2cm]{geometry}
\usepackage{fancyhdr,lastpage}
\usepackage{multicol}
\usepackage{hyperref}
\usepackage{amsmath}
\usepackage{amssymb}
\usepackage{mathrsfs}
\usepackage{extarrows}
\usepackage{algorithm}
\usepackage{algorithmic}

\usepackage{listings}
\lstset{
extendedchars=false,
language=python,
basicstyle=\ttfamily,
showstringspaces=false,
tabsize=2,
numbers=left,
numbersep=5pt,
frame=single,
framexleftmargin=0.8cm,
xleftmargin=0.8cm
}

\pagestyle{fancy}
\fancyhf{}
\rfoot{\scriptsize{\kaishu 第\thepage 页, 共\pageref{LastPage}页}}
\renewcommand\headrulewidth{0pt} % Removes funny header line
\usepackage{titlesec}
\titleformat{\section}{\Large\bfseries\flushleft} {{\bfseries\thesection\space}}{8pt}{}

\renewcommand{\theenumii}{\arabic{enumii}}
\renewcommand{\labelenumii}{(\theenumii)}
\newcommand{\id}[1]{\texttt{#1}}
\newcommand{\kw}[1]{\texttt{\textbf{#1}}}
%\newcommand{sgn}{{\operatorname{sgn}}}

\usepackage{enumitem}
\setenumerate[1]{itemsep=0pt}

\begin{document}

\title{一个Linux上用python直接处理csv文件的方法}
\author{\href{mailto:xtwxy@hotmail.com}{汪兴元}}
\maketitle
\abstract{
我们常常需要处理体积很大的csv数据或日志文件。一般来讲,
导入RDBMS来处理是常用的一种办法,但此办法也不完美,尤其是SQL这种描述型的语言。
在表达算法的时候不如一般的程序设计语言的表达力强,某些应用场合虽然能够实现,
但是难度太高,优化不易;批量处理大数据,通过在RDBMS上运行SQL效率并不高。

另一种办法就是使用Hadoop或Spark,
或导入NoSQL中,再使用MapReduce。但是使用类似Hadoop之类的工具又可能太重,
数据量不够运作一个Hadoop集群,还得承受其代价。

本文界绍了直接处理csv文件的一套办法,作为另一个数据处理的选项,示例代码用python,
运行的操作系统为Linux。
在从原始数据到最终的目标结果,需要组合本文提到的各种算法,才能达到目的。
通过管道-过滤器连接每一个算法,能够将整个处理算法分而治之,同时得到比较好的性能。

本文假设要处理的整个数据集无法一次装入机器的内存。
}

\tableofcontents

\section{校验数据}
csv文件可能存在错误。在正常情况下,csv文件不应存在错误。但是写入csv的过程并非
事务型的,不象RDBMS那样有很好的ACID保证,故磁盘耗尽,进程异常退出、挂起等因素,
直接导致csv文件的格式并非预期。

检查数据没有统一的方法,要视数据自身的特点来做检查。一般是检查一些数据正确
所需的必要条件,但必要并不一定充分。一般的检查方法有:
\begin{enumerate}
  \item 检查列数。如果列数不等于预期值,可以确定此行数据错误。
  \item 检查每列数据的格式,比如,数字,日期,或其它满足指定格式的文本串。
    可以偿试将其解析为对应的类型,看能否成功;或用正则表达式验证。
  \item 校验字段的值。前两项都对的情况下,可以检查数据值的取值范围。
    比如健在的人的年龄不能是负数,也不能是数百以上;历史数据中的记录时间不
    可能超过当前的日历时钟。 
\end{enumerate}
\lstinputlisting{python-src/validate_history_ai.py}
如果数据是已经通过有严格校验的系统中生成或导出的,原始数据本身无误,
一般做以上三项检查可以查出
我们能见到的全部错误。但是如果数据是人工填写或是有故障的机器生成的,
这三项检查仍不能确保检查出全部错误。

对于错误的数据该如何处理,要视情况而定。有的业务场景,例如Web服务器的
access log,价值不高;又如水温传感器输入的每分钟
一次的历史数据,可以丢弃,之后使用插补法补缺;有的场景是不可以这样做的,
如交易记录,需要重新导出此段数据或人工处理。样例代码中我们采用了丢弃的方法处理。

以上代码没有直接打开csv文件,而是从标准输入中读取文件,处理完毕之后再写到标准
输出。这样的好处是便于通过管道过滤器连接多个处理进程,避免过高的耦合度。
本文所有的处理程序都使用管道过滤器连接,不再赘述。

程序中对校验2.和3.未实现,可以练习下。对于机器导出的数据,
做完1.可以去掉大部分错误。如果不打算实现2.和3.

可以将列数从命令行上读取,成为更通用的子程序。这个作为练习。

\section{选择列}
原始数据中有可能只有部分列是所要关心的,其它的与要解决的问题无关,可以丢弃。
因此要选择列,类似SQL语句中的\id{SELECT}所要办的事情。
样例程序如下所示,这里我们只对第1,3,4,5列感兴趣。
\lstinputlisting{python-src/select_history_ai.py}

可以将列下标从命令行输入,这样这个程序就成为一个通用的选择程序,而不是仅用于
示例的场景。这个作为练习。


\section{排序}
我们对数据做去重、转置、分组、聚集,都需要先对数据排序。因此排序是非常重要的
功能。不可或缺。

大数据的排序由于无法直接一次装入内存,故必须使用外部排序。
如果能将数据切成能用内部排序的小块,排好序之后,再合并,
那么我们就可以完成对大数据的排序。

我们不打算从头实现一个大数据排序工具。利用Linux的工具

\section{转换ID字段}
\section{去重复值,\id{distinct()}}
\section{连接}
\subsection{内连接}
\subsection{左连接}
\subsection{全外连接}
\section{过滤数据}
\subsection{谓词及表示方法}
\subsection{谓词的连接关系}
\subsubsection{AND}
\subsubsection{OR}
\subsubsection{优先级}
\section{聚集}
\subsection{\id{max()}, \id{min()}, \id{avg()}}
\subsection{\kw{group by}, \kw{having}}
\subsection{\kw{having}}
\section{补缺失值}

\section{转置}

\end{document}

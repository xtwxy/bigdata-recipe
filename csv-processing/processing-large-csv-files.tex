\documentclass[12pt]{article}
\usepackage[fntef,hyperref,UTF8]{ctexcap}
\usepackage[a4paper, margin=2cm]{geometry}
\usepackage{fancyhdr,lastpage}
\usepackage{multicol}
\usepackage{hyperref}
\usepackage{amsmath}
\usepackage{amssymb}
\usepackage{mathrsfs}
\usepackage{extarrows}
\usepackage{algorithm}
\usepackage{algorithmic}

\usepackage{listings}
\lstset{
extendedchars=false,
language=python,
basicstyle=\ttfamily,
showstringspaces=false,
tabsize=4,
numbers=left,
numbersep=5pt,
frame=single,
framexleftmargin=0.8cm,
xleftmargin=0.8cm
}

\pagestyle{fancy}
\fancyhf{}
\rfoot{\scriptsize{\kaishu 第\thepage 页, 共\pageref{LastPage}页}}
\renewcommand\headrulewidth{0pt} % Removes funny header line
\usepackage{titlesec}
\titleformat{\section}{\Large\bfseries\flushleft} {{\bfseries\thesection\space}}{8pt}{}

\renewcommand{\theenumii}{\arabic{enumii}}
\renewcommand{\labelenumii}{(\theenumii)}
\newcommand{\id}[1]{\texttt{#1}}
\newcommand{\kw}[1]{\texttt{\textbf{#1}}}
%\newcommand{sgn}{{\operatorname{sgn}}}

\usepackage{enumitem}
\setenumerate[1]{itemsep=0pt}

\begin{document}

\title{一个直接处理csv文件的方法}
\author{\href{mailto:xtwxy@hotmail.com}{汪兴元}}
\maketitle
\abstract{
我们常常需要处理体积很大的csv数据或日志文件。一般来讲,
导入RDBMS来处理是常用的一种办法,但此办法也不完美,尤其是SQL这种描述型的语言。
在表达算法的时候不如一般的程序设计语言的表达力强,某些应用场合虽然能够实现,
但是难度太高,优化不易;批量处理大数据,通过在RDBMS上运行SQL效率并不高。

另一种办法就是使用Hadoop或Spark,
或导入NoSQL中,再使用MapReduce。但是使用类似Hadoop之类的工具又可能太重,
数据量不够运作一个Hadoop集群,还得承受其代价。

本文界绍了直接处理csv文件的一套办法,作为另一个数据处理的选项,示例代码用python。
在从原始数据到最终的目标结果,需要组合本文提到的各种算法,才能达到目的。
通过管道-过滤器连接每一个算法,能够将整个处理算法分而治之,同时得到比较好的性能。

本文假设要处理的整个数据集无法一次装入机器的内存。
}

\tableofcontents

\section{校验数据}
csv文件可能存在错误。在正常情况下,csv文件不应存在错误。但是写入csv的过程并非
事务型的,不象RDBMS那样有很好的ACID保证,故磁盘耗尽,进程异常退出、挂起等因素,
直接导致csv文件的格式并非预期。

检查数据没有统一的方法,要视数据自身的特点来做检查。一般是检查一些数据正确
所需的必要条件,但必要并不一定充分。一般的检查方法有:
\begin{enumerate}
  \item 检查列数。如果列数不等于预期值,可以确定此行数据错误。
  \item 检查每列数据的格式,比如,数字,日期,或其它满足指定格式的文本串。
    可以偿试将其解析为对应的类型,看能否成功;或用正则表达式验证。
  \item 校验字段的值。前两项都对的情况下,可以检查数据值的取值范围。
    比如健在的人的年龄不能是负数,也不能是数百以上;历史数据中的记录时间不
    可能超过当前的日历时钟。 
\end{enumerate}

如果数据是已经通过有严格校验的系统中生成或导出的,原始数据本身无误,
一般做以上三项检查可以查出
我们能见到的全部错误。但是如果数据是人工填写或是有故障的机器生成的,
这三项检查仍不能确保检查出全部错误。

\section{选择列}
\section{排序}
\section{去重复值}
\section{连接}
\subsection{内连接}
\subsection{左连接}
\subsection{全外连接}
\section{过滤数据}
\subsection{谓词及表示方法}
\subsection{谓词的连接关系}
\subsubsection{AND}
\subsubsection{OR}
\subsubsection{优先级}
\section{聚集}
\subsection{\id{max()}, \id{min()}, \id{avg()}}
\subsection{\kw{group by}, \kw{having}}
\subsection{\kw{having}}
\section{补缺失值}

\section{转置}

\end{document}

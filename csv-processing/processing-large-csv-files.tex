\documentclass[11pt]{article}
\usepackage[fntef,hyperref,UTF8]{ctexcap}
\usepackage[a4paper, margin=2cm]{geometry}
\usepackage{fancyhdr,lastpage}
\usepackage{multicol}
\usepackage{hyperref}
\usepackage{amsmath}
\usepackage{amssymb}
\usepackage{mathrsfs}
\usepackage{extarrows}
\usepackage{algorithm}
\usepackage{algorithmic}

\usepackage{listings}
\lstset{
extendedchars=false,
language=python,
basicstyle=\ttfamily,
keywordstyle=\bf,
showstringspaces=false,
tabsize=2,
numbers=left,
numbersep=5pt,
frame=single,
framexleftmargin=0.8cm,
xleftmargin=0.8cm
}

\pagestyle{fancy}
\fancyhf{}
\rfoot{\scriptsize{\kaishu 第\thepage 页, 共\pageref{LastPage}页}}
\renewcommand\headrulewidth{0pt} % Removes funny header line
\usepackage{titlesec}
\titleformat{\section}{\Large\bfseries\flushleft} {{\bfseries\thesection\space}}{8pt}{}

\renewcommand{\theenumii}{\arabic{enumii}}
\renewcommand{\labelenumii}{(\theenumii)}
\newcommand{\id}[1]{\texttt{#1}}
\newcommand{\kw}[1]{\texttt{\textbf{#1}}}
%\newcommand{sgn}{{\operatorname{sgn}}}

\usepackage{enumitem}
\setenumerate[1]{itemsep=0pt}

\begin{document}

\title{一个Linux上用python直接处理csv文件的方法}
\author{\href{mailto:xtwxy@hotmail.com}{汪兴元}}
\date{2015 年 7 月 21 日}
\maketitle
\abstract{
我们常常需要处理体积很大的csv数据或日志文件。一般来讲,
导入RDBMS来处理是常用的一种办法,但此办法也不完美,尤其是SQL这种描述型的语言。
在表达算法的时候不如一般的程序设计语言的表达力强,某些应用场合虽然能够实现,
但是难度太高,优化不易;批量处理大数据,通过在RDBMS上运行SQL效率并不高。

另一种办法就是使用Hadoop或Spark,
或导入NoSQL中,再使用MapReduce。但是使用类似Hadoop之类的工具又可能太重,
数据量不够运作一个Hadoop集群,还得承受其代价。

本文界绍了直接处理csv文件的一套办法,作为另一个数据处理的选项,示例代码用python,
运行的操作系统为Linux,源程序及csv数据文件的编码采用utf8。
在从原始数据到最终的目标结果,需要组合本文提到的各种算法,才能达到目的。
通过管道-过滤器连接每一个算法,能够将整个处理算法分而治之,同时得到比较好的性能。

本文假设要处理的整个数据集无法一次装入机器的内存。
}

\tableofcontents

\section{校验数据}
csv文件可能存在错误。在正常情况下,csv文件不应存在错误。但是写入csv的过程并非
事务型的,不象RDBMS那样有很好的ACID保证,故磁盘耗尽,进程异常退出、挂起等因素,
直接导致csv文件的格式并非预期。

检查数据没有统一的方法,要视数据自身的特点来做检查。一般是检查一些数据正确
所需的必要条件,但必要并不一定充分。一般的检查方法有:
\begin{enumerate}
  \item 检查列数。如果列数不等于预期值,可以确定此行数据错误。
  \item 检查每列数据的格式,比如,数字,日期,或其它满足指定格式的文本串。
    可以偿试将其解析为对应的类型,看能否成功;或用正则表达式验证。
  \item 校验字段的值。前两项都对的情况下,可以检查数据值的取值范围。
    比如健在的人的年龄不能是负数,也不能是数百以上;历史数据中的记录时间不
    可能超过当前的日历时钟。 
\end{enumerate}
\lstinputlisting{python-src/validate_history_ai.py}
如果数据是已经通过有严格校验的系统中生成或导出的,原始数据本身无误,
一般做以上三项检查可以查出
我们能见到的全部错误。但是如果数据是人工填写或是有故障的机器生成的,
这三项检查仍不能确保检查出全部错误。

对于错误的数据该如何处理,要视情况而定。有的业务场景,例如Web服务器的
access log,价值不高;又如水温传感器输入的每分钟
一次的历史数据,可以丢弃,之后使用插补法补缺;有的场景是不可以这样做的,
如交易记录,需要重新导出此段数据或人工处理。样例代码中我们采用了丢弃的方法处理。

以上代码没有直接打开csv文件,而是从标准输入中读取文件,处理完毕之后再写到标准
输出。这样的好处是便于通过管道过滤器连接多个处理进程,避免过高的耦合度。
本文所有的处理程序都使用管道过滤器连接,不再赘述。

程序中对校验2.和3.未实现,可以练习下。对于机器导出的数据,
做完1.可以去掉大部分错误。如果不打算实现2.和3.

可以将列数从命令行上读取,成为更通用的子程序。这个作为练习。

\section{选择列}
原始数据中有可能只有部分列是所要关心的,其它的与要解决的问题无关,可以丢弃。
因此要选择列,类似SQL语句中的\id{SELECT}所要办的事情。
样例程序如下所示,这里我们只对第1,3,4,5列感兴趣。
\lstinputlisting{python-src/select_history_ai.py}

可以将列下标从命令行输入,这样这个程序就成为一个通用的选择程序,而不是仅用于
示例的场景。这个作为练习。


\section{排序} \label{sec-sort}
我们对数据做去重、转置、分组、聚集,都需要先对数据排序。因此排序是非常重要的
功能。不可或缺。

大数据的排序由于无法直接一次装入内存,故必须使用外部排序。
如果能将数据切成能用内部排序的小块,排好序之后,再合并,
那么我们就可以完成对大数据的排序。

我们不打算从头实现一个大数据排序工具。利用Linux的工具\id{sort}来做排序。
先将csv文件拆分。可以在导出或生成csv的时候,每达到某一固定尺寸就rotate一下,
当前文件改名,再创建一个新文件当作当前文件。

假设文件已经切成能用内存装下的多个小文件。注意,使用切割工具切文件,
必须从整行数据的位置断开,否则,切口处的那条数据被切坏了。

对每一个文件,执行\id{sort}。下面的两段代码是在Linux的终端上输入的命令,“\id{\$}”表示
命令提示符。示例中有两个文件,这里只写了第一个文件的排序,另一个省略。
\begin{lstlisting}[language=sh]
$ cat data/att-rec-utf8_part-1.txt \ 
  | python python-src/select_att_rec.py \ 
  | sort --field-separator=, --ignore-leading-blanks --stable \ 
  --key=1.2n,2 --key=2,3 >part-1.csv
\end{lstlisting}
此命令先用\id{select\_att\_rec.py}选择所要的列,再对选择的结果排序,之后将结果重定向
到文件\id{part-1.csv}。所有小文件都排序好之后,再对排序后的文件做merge:
\begin{lstlisting}[language=sh]
$ sort --field-separator=, --ignore-leading-blanks --stable \
  --key=1.2n,2 --key=2,3 -m part-*.csv > merge-sorted.csv
\end{lstlisting}
注意在做merge之前,必须先对每个文件排好序;merge时,使用的关键字、分隔符等参
数必须与\id{sort}的时候所用的参数完全一致。

merge完之后,得到一个已经排好序的大文件。

\section{转换ID字段} \label{sec-transform-id}
一般在集成两个系统的数据的时候,需要将两个系统的数据合并起来。但是两个系统的
数据ID一般是相互独立分配的,因此,如果ID都是同一种类型,比如整数,几乎100\%
会发生冲突,即不相同的两个对象,其ID相同。即使用字符串之类的做ID,也难以不发
生冲突。因此必须要建立两个系统间的对象的对应关系。

例如,商品销售网站和库管理系统如果是两套独立的系统,分别由不同的团队开发,
比如库存管理系统是外购的,商品销售网站是自建的,如果要将两套系统的数据整
起来合做分析,又没有办法修改系统达到统一,就需要做ID转换。

下面的程序从标准输入读入历史数据,从命令行读取基础数据文件名,加载基础数据到
字典中,利用基础数据中的$R(\mathrm{ID}_{src}, \mathrm{ID}_{dest})$关系,做了一个映射。逐行将
原ID映射到目标ID的方法替换掉ID,实现ID转换的目的。
\lstinputlisting{python-src/transform_data_id.py}

一般来讲,大多数系统中的某一基础数据对象的ID是可以一次全部装入内存的。比如超市
的商品ID,虽然可能有数十万条,但是对于现代计算机的内存来讲并不算多。

对于历史记录,其身的ID,或Primary Key的数量大,其总体积有可能单台机器的内存装
不下。不过一般情况下,需要做转换的必要性往往不大,
只对其通过Foreign Key引用的基础数据对象的ID有转换的价值。如果实在需要转换,
使用\ref{sec-inner-join}提到的内连接方法来做大表的连接,再选择所要的字段,包括新ID,
不选原ID字段即可。

这个方法除了可以用于转换ID的场合,还可以用于更换名称。比如,我要将公司员工的
行为的分析结果公布,但是不想让人知道员工的真实姓名或工号;或者谁购买了伟哥,
将要商品名换掉。

样例程序中的ID列号、多少列都是hardcode进代码的。可以将关系搞成动态的,从命令行
读入。设$C_{i_1}, \cdots, C_{i_m}$是从要转换的数据中选出来作为原ID的列,
$C_{j_1}, \cdots, C_{j_n}$是要转成的目标ID列,$m, n \geqslant 1$。
则转码可以视为从
\begin{equation} \label{transform-id}
\begin{aligned} 
transform: R_{src} &\to R_{dest} \\
\mathrm{ID}_{src}(C_{i_1}, \cdots, C_{i_m}) &\mapsto \mathrm{ID}_{dest}(C_{j_1}, \cdots, C_{j_n}) 
\end{aligned} 
\end{equation}
只需要在命令行上输入源ID的列号清单和目标ID的列号清单,
在程序中,用迭代清单代替硬编码的取列语句即可。将hardcode转为动态,作为练习。

\section{去重复值,整行\id{distinct()}} \label{sec-remove-duplicate}
如果在内存中,用\id{set}来装数据,自动去重。但是我们的数据量大,不能装入内存,
因此直接用\id{set}不行。如果对每个输入值,都去扫描一遍记录,如果共有$n$个记录,
则对第$1$个记录,要扫描第$2, \cdots, n$共$n-1$个记录;
对第$2$个记录,要扫描第$3, \cdots, n$共$n-2$个记录,
对第$n-1$个记录要扫描第$n$个记录共$1$个记录。于是
一共要扫描文件$n-1$次,转储到文件$n-1$次,其中最后一次是结果,
中间$n-2$次是临时文件。这个性能相当差。共扫描记录的个次是
\begin{equation} \label{scan-identical}
\begin{aligned}
(n-1) + \cdots + 1 = \frac{1}{2} n (n-1)\text{。} 
\end{aligned} 
\end{equation}
csv文件每行的长度不固定,因此,定位记录的位置很难,往往扫描的记录数是$n^2$。

如果将数据先排序,则相同的两条记录会相邻,成为一组,于是,我们只需要比较相邻
的两条数据是否相同,如果相同就丢弃一条,如果不同就输出这一条。示例代码如下,
假设数据事先已经排好序。
\lstinputlisting{python-src/remove_duplicate.py}

\section{连接}
数据仅代表同一个表或关系的csv是不够的。两个表csv连接跟写SQL查询一样常见。
如果是历史记录表跟基础数据表的连接,一般基础数据很小,历史数据大,所以可以
采用~\ref{sec-transform-id}所述的方法。即把历史表的Key作为\id{dict}的Key,
基础表的Key作为\id{dict}的Value,逐行转换。

如果是内连接,则对于每行历史记录,必须找到对应的基础表中的记录才算连接上,
否则要丢弃;而对于左连接,找不到对应的记录,则用空值填充。

对于右连接,和全外连接,由于历史数据很大,不能直接装入内存,这办法不适用。

接下来描述不将表装入内存的连接方法。

\subsection{内连接} \label{sec-inner-join}
先将两个表的csv按连接的关键字排序,升序或降序都行,但必须都是升序或都是降序。

再逐行连接。不妨假设都按升序排好了序。设两个表分别为$R_1$,$R_2$,
当前从这两张表取的数据行的行$r_1$,$r_2$。

\begin{enumerate}
  \item 先从$R_1$中取第一行为$r_1$,取其连接的Key,设为$K_1$,
  取$R_2$中的第一行为$r_2$,取其Key,$K_2$,进行
比较。如果$K_1 < K_2$,由于$R_2$为升序,$r_2$及$R_2$余下的行的Key均不小于$K_2$,
因此,$r_1$与$r_2$及$R_2$余下任一行都连接不上。于是再取$R_1$中的
下一行为$r_1$。

  \item 如果$K_1 > K_2$,由于$R_1$升序,$r_1$及$R_1$余下的行的Key均不小于$K_1$,
  因此,$r_2$与$r_1$及$R_1$余下任一行都连接不上。于是再取$R_2$中的下一行为$r_2$。

  \item 如果$K_1 = K_2$,则连接成功,输出连接后的行。再取$R_2$的下一行为$r_2$。
重复以上步骤,直至所有的数据处理完毕。
\end{enumerate}
此算法的python实现如下。
\lstinputlisting{python-src/inner_join.py}

此算法利用数据集的顺序,避免了重复扫描csv,只需one pass即可完成连接;
缺点是必须先排序。

可以考虑将程序写成通用的,在命令行上指定连接谓词。这个可以作为练习。

\subsection{左连接}
同前述内连接,但是步情况1.连接不上$r_2$的时候,将$r_1$补齐空列之后输出。

\subsection{右连接}
同前述左连接,但是将$R_1$、$R_2$的位置互换即可。

\subsection{全外连接}
可以有两种办法可选:
\begin{enumerate}
  \item $R_1$、$R_2$先做一遍左连接,$R_1$、$R_2$再做一遍右连接,合并两个连接的结果,
最后去掉重复值即可。注意要保持左连接和右连接的列顺序是相同的。

 \item $R_1$、$R_2$先做左连接,再将结果与$R_2$做一个类似右连接的操作,
区别于右连接的是:对于有匹配的行,直接丢弃;对于无匹配的行,填写空值后输出。
\end{enumerate}
办法1.简单,但需要的步骤多,每个步骤都要用到文件;办法2.稍复杂,但步骤少
可以只写一次文件,采用管道直接连接中间结果。

从性能上考虑,办法2. 较优。

\section{过滤数据} \label{sec-filtering}
筛选满足条件的数据是信息检索中必不可少的部分。将用户的Search Criteria转成过滤
数据的Predicate。Search Criteria可能包含多个条件,为了简化问题,需要将每个条件
做成一个Predicate,只做一项测试,多个条件需要多个Predicate,并且每个Predicate
之间可能是AND,也可能是OR,并且有优先级顺序要求。

\subsection{谓词及表示方法}
谓词是一个返回Boolean值的表达式,用于测试对象的性质是否为真,或对象之间的关系
是否为真。

例如测试订单是否用银行卡支付;帐户余额是否大于100块;$a$同学的考试分数是否比$b$
同学的高,等等。
\begin{lstlisting}
for row in input:
  if predicate.test(row):
    output.writerow(row)
\end{lstlisting}
如果只测试其中的某列或几列,也可以如此:
\begin{lstlisting}
for row in input:
  if predicate.test(tuple([row[0], row[2]])):
    output.writerow(row)
\end{lstlisting}

经验不多的开发者在写过滤条件时,往往不注意将各谓词分解为基本的形式,并且与其它过滤条件,
可能还与过滤数据无关的问题,合并一起解决。这样一来,大片的其它逻辑与谓词测试的内容混
合在一起,修改起来要同时考虑多个问题。

这样在问题不太复杂的时候问题不大,而且可能性能比分离关注点来考虑更简单。
在解决一个较大的问题的时候,往往复杂到做不下去。

例如,使用的“Predicate”并不返回Boolean值,而是直接返回满足测试条件的数据,
并且做了处理,列数,字段的含义都可能已经变了,
因为恰好下一步也需要这个数据,这是常见的问题之一。例如,
\begin{lstlisting}
for row in input:
  result = predicate.test(row):
  if result != None:
    output.writerow(result)
\end{lstlisting}
上面的代码中,\id{result == row}未必能成立。光看这段代码,
天知道\id{predicate.test(row)}里面做了什么。

做Predicate的时候要把握以下原则,这些原则一般情况下应当尊守:
\begin{enumerate}
  \item Predicate只测试数据是否满足所指定的条件,满足反回True,不满足返回False。
    不能更改输入的数据。

  \item 不做与指定的测试无关的任何其它工作,例如,提取数据,计算平均值之类。
    如果所需要做的其它工作与测试结果有关,应交给Predicate的使用者来完成;
    如果无关,则更应分开来做,既不能在Predicate里做,又不能在Predicate的使用
    者里面做。

  \item Predicate不能有状态。其返回值仅取决于输入的条件,包括数据和测试条件。
    重复测试同一条数据,得到的结果应相同。
    如果使用lambda或闭包,其中捕获的参数不应发生变化。
 
\end{enumerate}

下面对这3条举例说明。

{\bf 原则1}. 可能出于下一步的数据处理方便的目的,在当前Predicate中对输入数据作调整,
容易实现,又比较方便。这样做,从名义上,Predicate只做测试,而不作修改,
而事实上却做修改,容易误导代码的读者;而且关注点不分离,引起复杂度上升,
例如,如果下阶段增加了一个Predicate需要没有修改的数据,这样就有矛盾了:
如果去掉原Predicate对数据的更改,则后面的处理不能正常运行,也要改。

{\bf 原则2}. 例如,有人做了一个比较大小的Predicate,用于扫描csv文件是否为升序,
通过比较第$n$条和第$n+1$条数据的大小。因此,可以顺便用来统计csv文件的行数。
假设数据满足升序条件,对每一条数据都做了扫描。

如果csv文件总共有$N$条数据,Predicate将会被调用$N-1$次。反过来,如果Predicate
被调用了$N-1$次,那么csv文件的行数应是$N$行。

假设文件既不满足升序,又不满足降序,那么在中间某一位置$k$,Predicate返回False,测试
终止,这时候Predicate统计的行数只能是前面有$k+1$条数据满足条件,这个与行数不符。

另有一排序功能需要用到比较的Predicate,于是该Predicate直接被用于排序,但排序
会用到多少次比较与数据的条数的关系不是这样的,于是这个行数统计毫无意义。

{\bf 原则3}. 例如比较大小。如果$a > b$,第一次测试\id{greaterPred.test(a, b) == True}
应能成立,而且无论做多少次测试,只要$a, b$保持原值,
\id{greaterPred.test(a, b) == True}必须都成立。

如果使用闭包,捕获的参数可能是引用一个闭包外的对象。由于这个对象不是常量,
且有并发访问,这个对象的状态可能会被修改,引起闭包的状态也随之变更。
这样一来,闭包的Test Criteria就变了,使用同样的数据做测试,返回值可能会与
改变前不同。
    
谓词分两种类型的:Unary和Binary。前者只需要输入一条数据,后者需要两条。
其它类型的可以通过这两类组合运算,或者将数据整合得到。
\begin{lstlisting}
def LE(a, b):
  return a <= b

pred = lambda x : LE(x, 10)

for row in inputFile:
  if pred(row):
    writer.writerow(row)
\end{lstlisting}

测试一条数据中的某一属性是否为真,可以用Unary。某些情况看似需要两个数据输入的,
也可以用Unary,但是使用lambda或闭包。例如,测试一个数是否大于某个常数,
需要比较两个数,但是常数可以使用lambda和一个\id{LE()}函数来捕获,这样,
得到的lambda对象只需要输入一个数。而对于检查顺序是否升序这种情况,
要比较的两个对象没有一个常量,因此直接用二元的\id{LE()}函数。
\begin{lstlisting}
def isAscending(l):
  n = None
  for i in l:
    if n is not None:
      if not LE(n, i):
        return False
    n = i
  return True    
\end{lstlisting}

以上代码有几个问题:假定每一行数据都不能是空行;有一个状态量\id{n},使得
程序的分支数比较多。如果这样写,分支总数没有变,但是每个子程序的分支数少了:
\begin{lstlisting}
class LEPreviousPred():
  def __init__(self):
    self.prev = None
  def test(x):
    old = self.prev
    self.prev = x
    if self.prev is not None:
      return LE(old, x)
    return False
      
def isAscending(l):
  pred = LEPreviousPred()
  lePrevious = lambda x : pred.test(x)
  for i in l:
    if not lePrevious(i):
      return False
    n = i
  return True    
\end{lstlisting}

就这个例子而言,除了\id{isAscending()}函数简化了一丁点,
这个做法总代码行数还多了,不划算。但是这个代码演示了将Predicate的测试逻辑与
调用者分离的一种办法。

在有较多Predicate一起参与的时候,程序逻辑势必复杂,这样一来,问题分成3份:

\begin{enumerate}
  \item 单个基本Predicate本身的实现。这个与实际要做的过滤程序程序的复杂度没有关系。
  \item 使用Predicate的过滤程序。这个程序只有一个单条件的\kw{if}测试。
  \item 多个基本的Predicate的组合,成为整个过滤子程序的Predicate,
  也就是2.中单条件\kw{if}语句所要测试的终极条件。
\end{enumerate}

而且此代码与前面原则3.不符。类似还有与时钟相关的测试,例如检测某事件发生的时
刻与当前时间的差距是否在某一范围内。由于时钟一直在走,必须每次都要取当前时间,
而其它的测试不必,这样在形式上不能统一,需要增加处理的复杂度。如果将取时钟
的动作放在Predicate内,则只需要将数据行送到Predicate作测试即可,成为一元Predicate。

这类测试的Predicate可以做成满足或不满足
原则3.形式。如果能大幅度简化程序或提高性能,并且不降底可读性的情况下可以
考虑不遵守,但是如果没有换取足够的好处,不应违反原则。

对于Predicate的接口形式,只要使用起来方便即可,使用lambda、普通函数、
类实例都可以。如果过滤数据的时候用到了很多个
Predicate,每个Predicate最好按类型统一接口,这样形式上保持一致,
可以用一致的方法处理;这些Predicate最好组合成一个Predicate,
能一次做完全部的测试,以便调用者在使用时,
不必关注Predicate之间的关系细节,只需关注它满足或不满足条件。

同时使用多个Predicate可以减少扫描文件的次数,但是,是否要分几次扫描文件,
要视情况而定。以下情况,如果用到Predicate,最好分步骤,一次只做一件事;
而且,不能与纯粹的条件过滤一起做。一般条件过滤只测试一行数据,且不改变这行数据。

\begin{enumerate}
  \item 去重复值。需要一次处理两行。
  \item 聚集。需要按Group处理,Group内都是多行。
  \item 根据输入列,计算其他列的值。输出的行与输入行不一样。
\end{enumerate}

主要问题是一起做会大幅度增加代码的复杂度,使得可读性、可维护性降底。


\subsection{谓词的连接关系}
前面已经提到Predicate的表示及如何使用。实际的数据过滤程序中,很少用到单一
Predicate。最基本的组合关系就是AND和OR加上优先级。

组合关系,本身就是
另一重Predicate。比如在$n$个Predicate $P_1, \cdots, P_n$中,至少有$m$个为True,
则Predicate $P$为True。因此,除测试$P_i \in {P_1, \cdots, P_n}$以外,还要
测试为True的$P_i$的个数是否不小于$m$。此类Predicate层出不穷,需要随业务需求而定,
不能一一列举,我们只列最基本的形式。

\subsubsection{AND}
必须每个Predicate返回True才算满足条件。直接使用编程语言的“与”运算符的版本不举例。
下面是支持多个Predicate的示例代码,注意如果\id{predicates}
是空列表的时候,所有数据都匹配,要考查这是否为预期的行为。
\begin{lstlisting}
for row in inputFile:
  match = True
  for p in predicates:
    if not p(row):
      match = False
      break
  if match:
    writer.writerow(row)
\end{lstlisting}
但是如果我们的Predicate并不都是用AND来连接的,那么这办法处理AND就不太好。
如果我们不加处理地直接使用Predicate,又要处理好它们的关系,
我们需要复杂的逻辑结构来利用这些Predicate,而且越复杂,越难实现,越难修改。

最好的办法是不影响过滤程序的结构,只改Predicate及其关系。我们将上面的代码改成
下面的形式:
\begin{lstlisting}
class AND():
  def __init__(self, predicates):
    self.predicates = predicates

  def test(row):
    for p in self.predicates:
      if not p(row):
        return False
    return True

andAll = AND(predList)
andAllPred = lambda x : andAll.test(x)

for row in inputFile:
  if andAllPred(row):
    writer.writerow(row)
\end{lstlisting}
主程序就仅剩第$14 \sim 16$行这三行。不管我们的条件如何,程序的结构应是如此。

是直接使用编程语言的“与”运算符,还是使用一个Predicate列表,要视复杂程度而定。
如果条件不多,应直接使用编程语言的逻辑运算符,或部分使用,
避免过度工程,吃力不讨好;
如果逻辑复杂,则全部使用本小节的办法。
分寸的拿捏要看实现的难度和程序的可读性,在两者之间取得平衡。

\subsubsection{OR}

类似AND连接,OR的直接实现代码如下:
\begin{lstlisting}
for row in inputFile:
  match = False
  for p in predicates:
    if p(row):
      match = True
      break
  if match:
    writer.writerow(row)
\end{lstlisting}

同样,如果要解耦Predicate和过滤程序的关系,代码如下:
\begin{lstlisting}
class OR():
  def __init__(self, predicates):
    self.predicates = predicates

  def test(row):
    for p in self.predicates:
      if p(row):
        return True
    return False

orAll = OR(predList)
orAllPred = lambda x : orAll.test(x)

for row in inputFile:
  if orAllPred(row):
    writer.writerow(row)
\end{lstlisting}

\subsubsection{优先级}
从前面的利用AND和OR组合Predicate的例子,我们知道组合后的结果可以表达为一个
Predicate。同理,优先级高的Predicate先运算,可以将它们做成一个Predicate再参与
运算。如果要求当前的表达式的值,参与当前表达式运算的Predicate或表达式必须先求
值,于是,优先级就实现了。
例如,我们要实现$(A \land B) \lor (C \land D) $,可以用下面的代码:
\begin{lstlisting}
andAB =  AND([A, B])
andCD =  AND([C, D])
andABPred = lambda x : andAB.test(x)
andCDPred = lambda x : andCD.test(x)

orAll = OR([andABPred, andCDPred])
orAllPred = lambda x : orAll.test(x)

for row in inputFile:
  if orAllPred(row):
    writer.writerow(row)
\end{lstlisting}

\section{聚集}
关于骤集函数的定义,参考\href{https://en.wikipedia.org/wiki/Aggregate\_function}{https://en.wikipedia.org/wiki/Aggregate\_function}。

要理解聚集,必须先搞清楚分组。聚集一般是按分组的,例如,找出每个班中数学考试
分数最高的,就是按班级分组,找出每个班的最高分,有多少个班就有多少个最高分;
除非全班旷考,或者存在无人的班级---有这可能性?
如果不指定分组条件,将是全部学生中最高的,只有最多一个数值。

\subsection{\kw{group by}}
在SQL中的\kw{order by}子句一般出现在带有聚集运算的\kw{select}语句尾部。
一般写SQL的时候,如果没有使用聚集函数,又用到了\kw{group by}子句,只有选择的列与
\kw{group by}的列数相同的时候是没有问题的,其它选择会报错。

使用SQL语句\kw{group by}的结果未必是同时\kw{order by}那些\kw{group by}所指定的那些列的,
只保证每一个group内部的行都相邻。
如果不指定\kw{order by},RDBMS会使用最有效率的方法\kw{group by}。

然而RDBMS高效\kw{group by}的方法,我们在python代码中用起来不容易,不现实。
我们为了让同一group的行都在一起,我们先将数据按\kw{group by}列\kw{order by},即先排序,
用前面第~\ref{sec-sort}节提到的Linux的\id{sort}。

如果没有用聚集函数,排完序之后,要除掉重复行。
使用第~\ref{sec-remove-duplicate}节的方法去除重复行。

如果使用了聚集,则在使用聚集函数的同时,必须检查每行是否是同一group。
如果正在处理的行不属于当前group则应计算当前group的聚集并输出,
再用当前行重新开一个group作为当前group。
由于做了group检测,所以不必再做去重这一步。
\begin{lstlisting}
class IsInGroup():
  def __init__(self, colnumbers):
    self.colnumbers = colnumbers
    self.curRow = None  

  def test(row):
    prevRow = self.curRow
    self.curRow = row

    if prevRow is None:
      return False
      
    for colno in colnumbers:
      if not (prevRow[colno] == row[colno]):
        return False
    return True

groupByColumnNumbers = [0, 2]
isInGroup = IsInGroup(groupByColumnNumbers)
isInGroupTest = lambda x : isInGroup.test(x)    

for row in inputFile:
  if isInGroupTest(row):
    ...
\end{lstlisting}

\subsection{\kw{having}}
SQL语句的\kw{where}子句不能用于聚集函数,只能用于表或视图的列。
SQL可以用\kw{having}来对聚集列做过滤。

我们在处理csv文件的时候,没有必须做此区别的限制。我们可以在输出聚集的结果之前,
对结果做过滤,也可以单独设一过滤环节,用管道连接到聚集的输出。

过滤数据用第~\ref{sec-filtering}节所述的方法。

\subsection{\id{max()}}
按前面的\kw{group by}和\kw{having}的方法,实现\id{max()}函数的功能。示例代码如下:
\begin{lstlisting}
def getGroupByCols(row, colNumbers):
  groupCols = []
  for colno in colNumbers:
    groupCols.append(row[colno])
  return groupCols

col4max = 1
max = None
for row in inputFile:
  if isInGroupTest(row):
    if row[col4max] > max:
      max = row[col4max]
  else:
    outrow = getGroupByCols(row, groupByColumnNumbers)
    outrow.append(max)
    max = None
    writer.writerow(outrow)
\end{lstlisting}

可以将上面的代码重构为通用的\id{max()}子程序。变化点在\kw{group by}的列号表
和要\id{max()}的列号。作为练习。 

本文中余下的其它聚集函数可以用类似的办法实现。
\subsection{\id{min()}}
\subsection{\id{avg()}}
\subsection{\id{count()}}
\subsection{\id{sum()}}
%\subsection{\id{median()}}
%\subsection{\id{mode()}}

\section{补缺失值}

\section{转置}

\end{document}
